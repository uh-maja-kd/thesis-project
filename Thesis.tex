\documentclass[11pt,oneside]{uhthesis}
%\usepackage{subfigure}
\usepackage[linesnumbered,lined,titlenumbered,ruled]{algorithm2e}
\usepackage{amsmath}
\usepackage{amssymb}
\usepackage{amsbsy}
\usepackage{mathpazo}
\usepackage{float}
\usepackage{braket}
\usepackage{subcaption}

%\floatstyle{ruled}
%\restylefloat{table}

\renewcommand{\tablename}{Tabla}
%\dontprintsemicolon

\title{Propuesta Basada en Aprendizaje Profundo para la Extracci\'on de Información de Textos con Contenido M\'edico Escritos en Español} 
\author{Ernesto Quevedo Caballero. \newline Alejandro Rodr\'iguez P\'erez}
\advisor{MsC. Juan Pablo Consuegra Ayala \newline Lic. Rocío Cruz Linares}
\degree{Licenciado en Ciencia de la Computación}
\faculty{Facultad de Matemática y Computación}
\date{Mayo de 2020}
\logo{Graphics/uhlogo}
\makenomenclature

\renewcommand{\vec}[1]{\boldsymbol{#1}}
\newcommand{\diff}[1]{\ensuremath{\mathrm{d}#1}}

\begin{document}

\frontmatter
\maketitle

\include{FrontMatter/Dedication}
%===================================================================================
% Chapter: Agradecimientos
%===================================================================================
\chapter*{Agradecimientos}\label{chapter:thanks}
\addcontentsline{toc}{chapter}{Agradecimientos}
%===================================================================================
%===================================================================================
% Chapter: Opinión del Tutor
%===================================================================================
\chapter*{Opinión del Tutor}\label{chapter:supervisor_opinion}
\addcontentsline{toc}{chapter}{Opinión del Tutor}
%===================================================================================
%===================================================================================
% Chapter: Resumen
%===================================================================================
\chapter*{Resumen}\label{chapter:abstract}
\addcontentsline{toc}{chapter}{Resumen}
%===================================================================================
\include{FrontMatter/Contents}

\mainmatter
	
%===================================================================================
% Chapter: Introducción
%===================================================================================
\chapter*{Introducción}\label{chapter:introduction}
\addcontentsline{toc}{chapter}{Introducción}

En las últimas décadas ha habido un crecimiento explosivo en la generación y recolección de datos en forma de texto.
Sin embargo, el gran volumen de información y la estructura semántica poco unificada que poseen los documentos escritos en lenguaje natural, hacen imposible a los investigadores encontrar buenos resultados eficientemente.
Esto ha causado un gran interés por parte de la comunidad científica en desarrollar sistemas que asistan la transformación de textos en conocimiento útil.
En este dominio se ubica el área de la extracción automática de información en la que, a su vez, se sitúan los problemas de extracción de entidades, así como de relaciones que se establecen entre las mismas.

En la esfera de la salud, la extracción automática de información cobra particular importancia.
Cada año se publica una gran cantidad de trabajos con temas y contenido médicos.
Extraer de manera automatizada información de los mismos podría contribuir a la obtención de resultados no evidentes para los investigadores, que mejorarían potencialmente el diagnóstico y tratamiento de enfermedades complejas.

Debido a que el español es una lengua menos generalizada que el inglés en términos de recursos computacionales, no existen muchos sistemas de extracción automática de información disponibles en este idioma.
Sin embargo, y concretamente en el dominio médico y de la salud, la colección de documentos escritos en idioma español es amplia y bien reconocida.
El desarrollo de trabajos investigativos y esfuerzos para la construcción de tales sistemas para textos no estructurados escritos en este idioma, posibilitaría un mayor aprvechamiento de los recursos de información disponibles.

Múltiples desafíos en la esfera de la extracción de información han sido organizados a lo largo de los años.
Tres de ellos, particularmente orientados a textos con contenido médico en idioma español, fueron los eventos: \emph{eHealth Knowledge Discovery} (\textit{eHealth-KD}), propuesto en el Taller de Análisis Semántico en la SEPLN\footnote{Sociedad Española de Procesamiento de Lenguaje Natural}~(TASS) del año 2018~\cite{martinez2018overview} y otras dos tareas similares propuestas en el Forum de Evaluación de Lenguajes Ibéricos~(\textit{IberLEF} por su nombre en inglés), en sus ediciones correspondientes al año 2019~\cite{piad2019overview} y 2020~\cite{piad2020overview}.
Estos desafíos propusieron la resolución de dos problemas fundamentales:

\begin{enumerate}
	\item Extracción y clasificación de entidades.
	\item Extracción y clasificación de relaciones semánticas.
\end{enumerate}

El problema de la extracción y clasificación de entidades aparece formulado en la literatura con el nombre de Reconocimiento de Entidades Nombradas~(NER por sus siglas en inglés).
Se define como obtener, a partir de texto no estructurado en lenguaje natural, una lista de las secciones de dicho texto que contienen entidades~\cite{piad2019overview, nadeau2007survey}.
Las entidades se han definido en la literatura de distintas formas, dependiendo del contexto, dominio y corpus utilizado.
Por otra parte, el problema de la extracción y clasificación de relaciones es más amplio, y está orientado a determinar, a partir de un conjunto de relaciones predefinido, cuáles de ellas se establecen en las entidades reconocidas en una oración~\cite{piad2019overview, kumar2017survey}.

Los tres eventos citados fueron escenarios propicios para evaluar modelos orientados a resolver estos problemas, debido a que propusieron un sistema de anotación novedoso, y un corpus anotado para entrenar, validar y evaluar las propuestas~\cite{piad2019corpus}.
Las soluciones presentadas se basan fundamentalmente en modelos de aprendizaje profundo, adaptando arquitecturas descritas en la literatura a los escenarios específicos de los concursos, obteniendo resultados bastante satisfactorios en las tareas de extracción de entidades, no tanto así de relaciones.

La presente investigación surge como parte de la participación en estas competencias.
La problemática que se plantea es precisamente la extracción y clasificación de entidades a partir de un modelo de anotación específico, en textos de contenido médico en idioma español, de tal forma que sea generalizable a otros dominios.
La investigación permitirá evaluar si las técnicas de aprendizaje profundo empleadas son efectivas para la solución de dicha problemática.


\subsection*{Objetivos}

La investigación se plantea como objetivo el diseño y validación de una solución computacional para la extracción de información en textos escritos en idioma español de contenido médico, que sea generarizable a múltiples dominios.

Se proponen los siguientes objetivos específicos:

\begin{enumerate}
	\item Consultar literatura especializada para identificar las técnicas de extracción y clasifiación de entidades y relaciones predominantes en el estado del arte, tanto en textos de dominio general como de dominio específico. Asimilar contenido.
	
	\item Diseñar propuesta propia que permita extraer y clasificar las entidades y relaciones relevantes, dentro del marco de un sistema de anotación específico.
	
	\item Construir un prototipo computacional para comprobar la eficacia de la estrategia propuesta.
	
	\item Evaluar la propuesta en un marco experimental que permita compararla con otras en el estado del arte.
	
\end{enumerate}

\subsection*{Contribuciones}

El sistema de anotación propuesto en los eventos \textit{eHealth-KD} es un escenario para medir el estado del arte en los problemas de extracción de entidades y relaciones en textos escritos en idioma español.
Los sistemas propuestos hasta ahora todavía no llegan a resolver satisfactoriamente los problemas planteado en dicho corpus.
Luego, una contribución fundamental de este trabajo es la presentación de modelos competitivos en ambas subtareas del evento \textit{eHealth-KD 2020}, el cual reúne los resultados más relevantes obtenidos hasta ahora.

Otra contribución de la investigación lo constituye el estudio sistemático realizado sobre algunos de los factores que influyen en la efectividad del modelo presentado.

Particularmente, como parte de la propuesta para resolver el problema de extracción de entidades se introduce un algoritmo lineal para la reconstrucción de entidades a partir de secuencias de etiquetas en el sistema BMEWO-V, que tiene en cuenta la presencia de entidades solapadas y discontinuas.
Y, como parte del estudio del modelo de extracción de relaciones propuesto, se valida la utilidad de información derivada del árbol de dependencias de la oración en la resolución de esta tarea en el contexto del evento \textit{eHealth-KD 2020}.

\subsection*{Organización de la Tesis}

El contenido de la tesis se organiza de la siguiente forma.
El capítulo \ref{chapter:information_extraction} introduce los principales conceptos relacionados con la extracción de información y describe las principales técnicas que se han empleado en la literatura en distintos escenarios.
Además, se describe el modelo de anotación en el que se centra esta investigación.
Los capítulos \ref{chapter:entities} y \ref{chapter:relations} contienen las propuestas de solución para los problemas de extracción de entidades y relaciones respectivamente.
Finalmente, el capítulo \ref{chapter:experiments} describe el marco experimental al que fueron sometidas las propuestas, se muestran los resultados, y se discute la efectividad de cada uno de los modelos propuestos en la tesis en función de los resultados obtenidos.
La tesis finaliza presentando las conclusiones de la investigación y la recomendación de futuras direcciones de trabajo.

%===================================================================================
%===================================================================================
% Chapter: Extracción de Información
%===================================================================================
\chapter{Extracción de Información}\label{chapter:information_extraction}
\addcontentsline{toc}{chapter}{Extracción de Información}

\section{Extracción de Entidades}

\section{Extracción de Relaciones}

\section{Enfoque Conjunto}

Comúnmente, las tareas de NER y ER son abordadas secuencialmente, con sistemas que extraen entidades y luego determinan qué relaciones existen entre las mismas.
Sin embargo, sistemas secuenciales de este tipo son propensos a propagar el error de una tarea hacia la siguiente.
Y, en dependencia de cómo estén implementados, puede que no exploten información relevante el uno del otro.
Es por ello que existe toda una línea de desarrollo orientada a explorar soluciones que resuelvan simultáneamente estas dos tareas~\cite{miwa2016end, li2017neural, bekoulis2018adversarial, bekoulis2018joint, li2019entity, nguyen2019end, giorgi2019end}.

La propuesta general se basa en definir una arquitectura que tenga salidas que le permitan resolver los dos problemas, similares a las descritas en secciones anteriores.
Con la particularidad de que se define una función de pérdida que considere ambas tareas, permitiendo así optimizar el modelo en función de resolver ambos problemas.
Además, algunas de las capas que codifican la información de entrada de estos modelos son compartidas en el cómputo de cada salida, permitiendo así el uso de información de una tarea en la resolución de la otra.

Algunas de estas propuestas, al igual que otras ya descritas, se apoyan en recursos externos de NLP, como son los \textit{parsers} de dependencias~\cite{miwa2016end, li2017neural}.
Esto, como se ha explicado, limita la efectividad de estos modelos en dominios donde dichas herramientas no funcionan como se esperaría~(e.g en dominios en los que no fueron entrenadas).

En respuesta a esto, se han desarrollado propuestas puramente \textit{end-to-end}, que no se apoyan en estos recursos~\cite{bekoulis2018joint}.
Este trabajo fue extendido con entrenamiento adversarial~\cite{bekoulis2018adversarial}.
En 2019, \textit{Nguyen y Verspoor}~\cite{nguyen2019end} propusieron un modelo similar, que usa adicionalmente un mecanismo de \textit{biaffine attention}~\cite{biaffineattention}.
\textit{Giorgi et al, 2019}~\cite{giorgi2019end} propusieron un modelo similar que incluye el uso del modelo preentrenado del lenguaje BERT, aliviando el costo de entrenamiento.
Por su parte, \textit{Li et al, 2019}~\cite{li2019entity}, enfocaron el problema con \textit{multi-turn} question answering, mediante un modelo de QA basado en BERT que respondía preguntas cuyas respuestas eran las las entidades y relaciones entre las mismas.
%===================================================================================


%===================================================================================
% Chapter: Propuesta para la Extracción de Entidades
%===================================================================================
\chapter{Propuesta para la Extracción de Entidades}\label{chapter:entities}
\addcontentsline{toc}{chapter}{Extracción de Entidades}
%===================================================================================

%===================================================================================

En este capítulo se hace una descripción de las técnicas empleadas para la resolución de la tarea de Extracción de Entidades, donde se describe como funciona el modelo de aprendizaje profundo propuesto, la construcci\'on de su entrada a partir de una oraci\'on, entre ellas la obtenci\'on de los \emph{embeddings contextuales} de las palabras de una oraci\'on hacienqdo uso de un modelo preentrenado de BERT. Finalmente se describe cada una de las componentes del modelo de aprendizaje profundo y su funcionamiento.


\section{Modelo de Aprendizaje Profundo}

El modelo propuesto es un BiLSTM-CRF apilado que tiene como entrada \emph{embeddings contextuales} pre-entrenados con BERT, \emph{postag embeddings} que son entrenados junto al modelo y \emph{character embeddings} que se entrenan junto al modelo a partir de una CNN. El modelo tiene como decodificador para la predicci\'on de las etiquetas correspondientes a cada token un CRF. El modelo pretende determinar cu\'ales son las entidades dentro de la oraci\'on y de esas entidades cu\'al es su clasificaci\'on como \emph{Concepto}, \emph{Acci\'on}, \emph{Predicado} y \emph{Referencia}.

\subsection{Entrada del Modelo}\label{sec:entrance}
Se recibe como entrada una oraci\'on en texto plano, la cual necesita preprocesamiento para construir la entrada apropiada del modelo. El primer paso es tokenizar las oraciones dado que todas las entradas del modelo esperan una secuencia de tokens. Para esto se utiliza la tokenizaci\'on a nivel de palabras.

Por cada token en el cual una oraci\'on fue dividida, la entrada respectiva a ese token consiste de una lista de tres vectores de rasgos.

\begin{description}
	\item[Vector de PoS-tag:] Es un vector \emph{one-hot} de codificaci\'on de la informaci\'on de \emph{Part of Speech} (PoS).
	\item[Codificaci\'on de los caracteres:] Es la concatenaci\'on de los vectores \emph{one-hot} de codificaci\'on de cada uno de los caracteres contenidos en la palabra. 
	\item[Embedding Contextual:] Es un vector de \emph{embedding} de la palabra conformado por la concatenaci\'on de los vectores que representan a dicha palabra en cada una de las capas de BERT.
	 
\end{description} 

Para extraer la informaci\'on de \emph{PoS-tag} se utiliza la librer\'ia de python \texttt{spacy}~\footnote{spacy.io}. 

Para la construcci\'on de los \emph{emmbeddings contextuales} correspondientes a cada token se utiliza un modelo preentrenado de \textbf{BERT}, siguiendo el siguiente procedimiento. Se toma la oraci\'on y se le agregan al inicio y al final las cadenas de texto \emph{"[CLS]"} y \emph{"[SEP]"} respectivamente. Luego esta nueva oraci\'on es tokenizada utilizando un algoritmo de tokenizaci\'on de subpalabras conocido como \emph{Word Piece}~\cite{schuster2012japanese}. El vocabulario de \emph{Word Piece} de BERT se computa aplicando el algoritmo de \emph{Word Piece tokenization} en cada secuencia de caracteres del corpus en el que se entren\'o BERT: \emph{Wikipedia and the Book Corpus}, lo cual resulta en 30 mil tokens de vocabulario. Como es l\'ogico, debido al tipo de tokenizaci\'on \emph{Word Piece}, se quisiera distinguir dentro del vocabulario a las palabras \emph{venoso} como una sola palabra y el sufijo \emph{venoso}, por lo que el sufijo se representa de la forma \emph{\#\#venoso} en el vocabulario.

%por el \emph{BertTokenizer} de la biblitoeca \textbf{pytorch-pretrained-bert}~\footnote{https://pypi.org/project/pytorch-pretrained-bert/} utilizando un algoritmo de tokenizaci\'on de subpalabras conocido como \emph{Word Piece}~\cite{schuster2012japanese}. 

%La tokenizaci\'on de \emph{Word Piece} es en esencia la idea de un algoritmo de compresi\'on, en este caso de representar palabras frecuentes con menos s\'imbolos y palabras menos frecuentes con m\'as s\'imbolos lo cual es de hecho la idea que hay detr\'as de varios esquemas de codificaci\'on como la codificaci\'on de \emph{Huffman}. \emph{Word Piece aplica el mismo principio} y t\'ecnicas a la tokenizaci\'on. \emph{Word Piece} es un algoritmo \emph{bottom- up} para la tokenizaci\'on de subpalabras que aprende un vocabulario de subpalabras de cierto tama\~no. La idea b\'asica es la siguiente:
%
%\begin{enumerate}
%	\item Comienza separando todas las palabras en caracteres unicode. Cada caracter unicode corresponde a un s\'imbolo en el vocabulario final. Se comienza con este vocabulario minimalista y luego se expande.
%	\item Mientras a\'un haya espacio en el vocabulario se hace lo siguiente:
%	\begin{enumerate}
%		\item Encontrar la pareja (\emph{bigram}) de s\'imbolos tal que al mezclarla incrementa the likelihood de un modelo de lenguaje unigram entrenado en los datos de entrenamiento. Este tipo de mezcla adem\'as de la frecuencia de la pareja de s\'imbolos, tiene en cuenta la frecuencia de los s\'imbolos del vocabulario inicial que los conforman. La log likelihood de una oraci\'on en una modelo de lenguaj unigram (asumiendo independencia entre las palabras) es la suma de las frecuencias de los s\'imbolos que los componen. Esto significa que mezclar dos s\'imbolos incrementa el total de la log likelihood por el log likelihood de los dos s\'imbolos mezclados y disminuir por los log likelihod de los dos s\'imbolos originales. Asumiendo que se mezclan \emph{x} y \emph{y}, el incremento en el log likelihood total ser\'ia de:
%		
%		\begin{equation}
%			\log p(x,y) - \log p(x) - \log p(y) = \log \frac{\log(p(x))}{\log(p(x))\log(p(y))}
%		\end{equation}
%		
%		\item Mezclar esos dos s\'imbolos para crear un nuevo s\'imbolo y a\~nadirlo al vocabulario. Esto incrementa el tama~no del vocabulario en 1.
%	\end{enumerate}
%	
%\end{enumerate} 

Posteriormente los tokens obtenidos tras tokenizar la oraci\'on utilizando el \emph{Word Piece tokenizer} son llevados a \'indices con respecto al vocabulario de BERT. Adem\'as se contruye la m\'ascara de atenci\'on poniendo a todos los tokens el valor de \textbf{1}, dado que no es necesario para nuestros objetivos hacer \emph{padding} de las oraciones. Luego la secuencia de \'indices de los tokens en el vocabulario de BERT y la m\'ascara de atenci\'on son pasadas como entrada del modelo pre-entrenado de BERT, obteniendo as\'i como salida la codificaci\'on de cada una de las capas de BERT para cada una de las palabras.

Luego el \emph{embedding contextual} de cada token se construye a partir de la concatenaci\'on de la codificaci\'on de dicho token en las \textbf{n} capas de BERT, o sea, si el token $t$, tiene codificaciones $l_1, l_2, ..., l_n$ en cada una de las capas de BERT respectivamente, entonces el \emph{embedding contextual} del token $t$ ser\'ia la concatenaci\'on de cada una de las codificaciones anteriores, resultando en el vector: $l_1l_2....l_n$.

Sin embargo, debido a la tokenizaci\'on del tipo \emph{Word Piece} realizada, hay palabras que quedan separadas en varios tokens, como por ejemplo \emph{[c\'an], [\#\#cer]}, donde la palabra \emph{c\'ancer} es una entidad y se clasificar\'ia como \emph{Concepto}, pero para que la arquitectura del modelo de aprendizaje profundo pueda etiquetar correctamente a la palabra c\'ancer, tendr\'ia que etiquetar dos tokens en vez de uno, por eso para la entrada del modelo de aprendizaje profundo para resolver la \emph{Subtarea A} se utiliza el tokenizer a nivel de palabras y no el tokenizer de BERT. Por lo que queda el problema de que al obtener, del modelo pre-entrenado de BERT, los \emph{embeddings contextuales} de cada uno de los tokens obtenidos de la \emph{Word Piece tokenization} es necesario computar cu\'al ser\'ia el \emph{embedding contextual} de los tokens obtenidos por la tokenizaci\'on a nivel de palabras. Luego para lograr este objetivo se dise\~na un algoritmo de \textbf{mapping}. 

El algoritmo de \emph{mapping} consiste b\'asicamente en la siguiente idea. Sea \emph{w} un token obtenido por la tokenizaci\'on por espacios y basada en reglas y los tokens $t_1, t_2, ..., t_k$ tokens obtenidos por el tokenizer de BERT tales que:

\begin{equation}
	w = \sum_{i = 1}^{k} t_i
\end{equation} 


y sean $e_1, e_2, ..., e_k$ los \emph{embeddings contextuales} obtenidos a partir de BERT para cada uno de los tokens $t_i$ ($1 \leq  i \leq k$), entonces sea $e_t$ el  \emph{embedding contextual} correspondiente al token $t$, el mismo se obtendr\'ia de la siguiente forma:

\begin{equation}
	e_t = mean(e_1, e_2, ... , e_k)
\end{equation}

Donde \emph{mean} es la funci\'on que devuelve la media de los vectores $e_1, e_2, ..., e_k$. De esat forma se obtiene un \emph{mapping} de los \emph{embeddings contextuales} de BERT para los \emph{word piece tokens} hacia los tokens obtenidos a por la tokenizaci\'on a nivel de palabras.

\subsection{Arquitectura del Modelo}

El modelo recibe una secuencia $S$ de tokens como entrada tal y como fue descrito en~\ref{sec:entrance}. La arquitectura se divide en tres niveles: (1) representaci\'on de los \emph{tokens}, (2) respresentaci\'on de la secuencia, y (3) predicci\'on de las etiquetas \emph{BMEWO-V} y la clasificaci\'on seg\'un las etiquetas \emph{Concept}, \emph{Action}, \emph{Reference}, \emph{Predicate} y \emph{None}, esta \'ultima en caso de no pertenecer a ninguna de las anteriores. La figura~\ref{fig:ArcModRec} y~\ref{fig:ArcModClass} resumen la arquitectura del modelo.


\begin{figure}[h!]
	\centering
	\includegraphics[width = 10cm]{Graphics/EntitiesModelRec.png}
	\caption{Arquitectura del modelo de aprendizaje profundo para la extraccio\'on y clasificaci\'on de entidades con el CRF para la decodificaci\'on de las etiquetas \emph{BMEWO-V}.}\label{fig:ArcModRec}
\end{figure}

\begin{figure}[h!]
	\centering
	\includegraphics[width = 10cm]{Graphics/EntitiesModelClas.png}
	\caption{Arquitectura del modelo de aprendizaje profundo para la extraccio\'on y clasificaci\'on de entidades con el CRF para la decodificaci\'on de las etiquetas de clasificaci\'on \emph{Concept}, \emph{Action}, \emph{Reference}, \emph{Predicate} y \emph{None},}\label{fig:ArcModClass}
\end{figure}

En un primer nivel, el modelo se encarga de obtener la representaci\'on de cada token en la secuencia de entrada, para ello existe una capa intermedia que trabaja en funci\'on de transformar la caracter\'istica \emph{Char Indexes} en vectores, que junto al de \emph{PosTag} y el \emph{embedding contextual} forman la representaci\'on final del token. Una capa CNN convierte los \emph{Char Indexes} en un vector de \emph{embedding} que atrapa el significado sem\'antico del token a nivel de caracteres. Esto se incluye para que el modelo se auxilie de caracter\'isticas morfol\'ogicas de la palabra (como prefijos y sufijos) en caso de que el token no forme parte del vocabulario del modelo de BERT preentrenado. El vector obtenido a partir de la CNN concatenado con el vector de \emph{embedding contextual} y el de \emph{PosTag} forman la representaci\'on del token.

En un segundo nivel, el modelo procesa la secuencia de tokens para obtener representaciones a nivel de oraci\'on. Una capa BiLSTM recorre la secuencia de tokens en ambos sentidos para construir dos secuencias de vectores. Los vectores en posiciones complementarias de las dos secuencias son concatenados, obteni\'endose as\'i nuevamente una secuencia $P$ que asocia un vector dependiente del contexto, que captura informaci\'on de la oraci\'on completa, a cada token de la oraci\'on. Esta secuencia busca atrapar las dependencias semánticas entre los tokens de la oración.
Luego una segunda BiLSTM apilada, que se utiliza tradicionalmente para a\~nadir m\'as poder representacional, recorre la secuencia devuelta por la primera BiLSTM en ambos sentidos, para tambi\'en construir dos secuencias de vectores que son concatenadas, produciendo as\'i otra nueva secuencia $P'$ que asocia un vector a cada token de la oraci\'on capturando informaci\'on m\'as compleja que la obtenida por la primera capa BiLSTM.

\begin{equation}
	P = BiLSTM(S)~~~~~~~~P' = StackedBiLSTM(P)
\end{equation}

En el \'ultimo nivel, la secuencia $P'$ sirve de entrada para dos capas CRF que producen como salida la etiqueta mas probable. Sean $x_tag$ y $x_type$ las salidas corrspondientes al sistema de etiquetado \emph{BMEWO-V} y el tipo de entidad respectivamente; y sean $CRF_tag$ y $CRF_type$ las respectivas capas CRF, entonces:

\begin{equation}
	x_tag = CRF_tag(P')~~~~~~~x_type = CRF_type(P')
\end{equation}

\subsection{Postprocesamiento}
La primera capa de salida CRF produce una secuencia en el sistema de etiquetas $BMEWO$-$V$.
Las siglas responden a la nomenclatura: $B$ para inicio de una entidad, $M$ para palabras interiores, $E$ para palabras del final, $W$ para aquellas palabras que constituyen una entidad por sí mismas y O para las palabras que no forman parte de ninguna entidad.
También contempla la posibilidad de etiquetar palabras que pertecen a más de una entidad~(solapamiento de entidades), para lo cual incluye la etiqueta $V$.

Para extraer el conjunto de entidades presentes en la oración a partir de la salida de la primera capa CRF, se utiliza un proceso que será referido de ahora en adelante como \textbf{decodificación}.
La decodificación se complejiza debido a un elemento característico del problema descrito: las palabras pertenecientes a una misma entidad no aparecen necesariamente de manera contigua en la oración.
Por ejemplo, en la oración: \textit{La vitamina D también juega un rol en su sistema nervioso, muscular e inmunitario.}, extraída de la colección de entrenamiento del evento \textit{eHealth-KD 2019}, aparecen marcadas como relevantes las entidades \textit{sistema nervioso}, \textit{sistema muscular} y \textit{sistema inmunitario}; las dos últimas formadas por palabras no consecutivas en la oración.
Teniendo en cuenta esto, el proceso de decodificación se divide en dos etapas.
En una primera instancia se detectan y extraen el conjunto de entidades entre las que se encuentran todas aquellas que están formadas por palabras no contiguas~(entidades discontinuas); luego, se extraen todas aquellas formadas por \textit{tokens} adyacentes~(entidades continuas), tomando como premisa que no existe ninguna discontinua restante.

De acuerdo al uso correcto del idioma Español, se redujeron los escenarios donde pueden aparecer entidades discontinuas a las subsecuencias que se corresponden con las expresiones regulares: $(V^+)((M^*EO^*)+)(M^*E)$ y $((BO)^+)(B)(V^+)$.
La primera agrupa a las entidades que comparten los \textit{tokens} iniciales; la segunda, a aquellas que comparten las palabras finales.
Un ejemplo que se incluye en el primer grupo lo constituye el fragmento: \textit{cáncer de mama y de pulmón}, correctamente etiquetado como $VMEOME$, de donde se extraen las entidades \textit{cáncer de mama} y \textit{cáncer de pulmón}.
Como ejemplo del segundo grupo se puede citar el fragmento: \textit{tejidos y órganos humanos}, correctamente etiquetado como $BOBV$, de donde se extraen las entidades \textit{tejidos humanos} y \textit{órganos humanos}.
El proceso de decodificación de estos dos escenarios permite obtener satisfactoriamente las entidades discontinuas en las colecciones de entrenamiento y desarrollo con un error menor al $1\%$, asumiendo que las secuencias están etiquetadas correctamente.


Después de a detección de posibles entidades discontinuas, la segunda etapa comienza asumiendo que las restantes están formadas por secuencias continuas de palabras.
Para ello las etiquetas asociadas a todas las entidades extraídas en la primera etapa son cambiadas a $O$. 


Para extraer las entidades continuas se lleva a cabo un proceso iterativo sobre la secuencia de etiquetas.
Debido a las limitaciones del sistema $BMEWO$-$V$, también se asume que en una misma palabra no se solapan más de dos entidades.
Abandonar esta suposición no aporta mucha mayor información~\footnote{Cuando se dice que no aporta mucha mayor información se hace referencia al hecho de que no es común, de acuerdo a la evaluación que se realizó sobre las colecciones de entrenamiento y desarrollo, encontrar tres o más entidades solapadas en una misma palabra}, además de que se introduce ambigüedad en el proceso dada por la incertidumbre que habría en dicho caso con respecto al número de entidades a las que el \textit{token} en cuestión pertenecería.
Dado esto, a lo largo del procedimiento se matienen dos entidades en formación.
En cada iteración estas dos entidades son creadas, extendidas o emitidas de acuerdo con reglas definidas en una máquina que actúa de manera determinista en función solamente de la etiqueta actual y la anterior en la secuencia que se procesa.
La figura \ref{fig:automaton} resume la función de decisión de dicho autómata. 

\begin{figure}[h!]
	\centering
	\includegraphics[width = 10cm]{Graphics/automaton.png}
	\caption{Función de decisión del autómata en la segunda etapa de la decodificación de la salida del modelo de extracción de entidades.}\label{fig:automaton}
\end{figure}

De acuerdo a evaluaciones efectuadas en las colecciones de entrenamiento y desarrollo, el proceso de decodificación de secuencias etiquetadas de manera correcta, extrae adecuadamente más del $98\%$ de las entidades presentes.


Luego de identificadas las entidades, para clasificar cada una de ellas de acuerdo a su tipo, se utiliza un sistema de votaci\'on, basado en la salida de la segunda capa CRF.
La misma asocia, para cada palabra en la oración de entrada, uno de los tipos de entidad, en este caso uno entre: \texttt{Concept}, \texttt{Action}, \texttt{Predicate} o \texttt{Reference}.
Cada palabra produce para cada entidad a la que pertenece, un voto de acuerdo al tipo que le fue asociado.
Luego, cada entidad se clasifica de según el tipo que mayor cantidad de votos que haya obtenido. Si está equilibrada la votación se asume \texttt{Concepto}, por ser la más frecuente por amplio margen en las colecciones estudiadas.




















%===================================================================================
% Chapter: Propuesta para la Extracción de Relaciones
%===================================================================================
\chapter{Propuesta para la Extracción de Relaciones}\label{chapter:relations}
\addcontentsline{toc}{chapter}{Extracción de Relaciones}

En este capítulo se hace una descripción de las técnicas empleadas para la resolución de la tarea de Extracción de Relaciones.
Primeramente, se recoge un resumen del análisis de dependencias de una oración, aspecto cardinal en la propuesta realizada.
Luego se describe el modelo de aprendizaje profundo propuesto, sus componentes y sus distintas variantes.


\section{Análisis de Dependencias}\label{sec:parsing}

El conocimiento de la estructura y la sintáxis que subyace en un texto en lenguaje natural puede ser de mucha ayuda en tareas típicas de NLP como la clasificación de texto, la sumarización o la extracción de relaciones.
Una de las técnicas más comunes para capturar cierta estructura en las oraciones es el análisis sintáctico~(conocido comúnmente por su nombre en inglés: \emph{parsing}).
En esta sección se abordará este tema, particularmente el análisis de dependencias.

Hay dos formas de describir la estructura de una oración en lenguaje natural: separando la oración en \textbf{constituyentes}~(frases), que se separan a su vez en constituyentes más pequeños; o estableciendo conexiones entre las palabras individuales~\cite{covington2001fundamental}.
El significado de estas dos variantes se ilustra en las figuras \ref{fig:dep_const} y \ref{fig:dep_links} respectivamente.

\begin{figure}[h!]
	\centering
	\includegraphics[width=0.8\linewidth]{Graphics/dep_const.png}
	\caption{Estructura constituyente para una oración en idioma inglés del \emph{Penn Treebank}.}\label{fig:dep_const}
\end{figure}

\begin{figure}[h!]
	\centering
	\includegraphics[width=0.8\linewidth]{Graphics/dep_links.png}
	\caption{Estructura de dependencias para una oración en idioma inglés del \emph{Penn Treebank}.}\label{fig:dep_links}
\end{figure}

La representación constituyente del lenguaje data de varios años atrás, y ha sido explotada tanto por los científicos de la computación como por lingüistas, en aras de obtener buenas representaciones del lenguaje natural.
Sin embargo, la comunidad científica ha mostrado un creciente interés en los últimos años en las estructuras de dependencias como una alternativa a esta representación.

La noción fundamental de \textbf{dependencia} está basada en la idea de que la estructura sintáctica de una oración está conformada por un conjunto de relaciones binarias asimétricas entre las palabras de dicha oración~\cite{nivre2005dependency}.
Siempre que se establece una relación entre dos palabras, a una de ellas se le denomina \textbf{cabecera} y a la otra \textbf{dependiente}.
A continuación se listan algunos criterios que han sido propuestos para identificar una relación sintáctica entre una cabecera $H$ y un dependiente $D$, en una construcción sintáctica $C$~\cite{zwicky1985heads, richard1990english}:

\begin{enumerate}
	\item $H$ determina la categoría sintáctica de $C$, y muchas veces puede sustituir a $C$.
	
	\item $H$ determina la categoría semántica de $C$, mientras que $D$ aporta especificidad.
	
	\item $H$ es obligatoria, mientra que $D$ es opcional.
	
	\item $H$ selecciona a $D$, y determina si puede o no ser opcional.
	
	\item La forma de $D$ depende de $H$.
	
	\item La posición de $D$ en la oración se especifica con respecto a $H$.
\end{enumerate}

Estas reglas no son absolutas y contienen un mezcla de criterios variados, algunos sintácticos, otros semánticos.
No existe en la literatura una noción coherente de dependencia que se corresponda con todos los distintos criterios~\cite{nivre2005dependency}.


\subsection{Grafo de dependencias}

Si se considera cada dependencia como un arco dirigido que tiene como origen a la cabecera y como destino al dependiente, la estructura de dependecias de la oración conforma un grafo dirigido $G$ cuyos nodos son los elementos léxicos del lenguaje~(\emph{tokens}).
Además, el grafo subyacente de $G$ debe estar conectado, para que cada nodo esté relacionado con, al menos, otro nodo.

A esta caracterización se le imponen usualmente más restricciones.
Dos de las más utilizadas en las distintas formalizaciones de gramáticas basadas en dependencias~(o simplemente, gramáticas de dependencias), son: la suposición de que cada nodo del grafo tiene \emph{indegree} $\leq 1$; y la no existencia de ciclos.
Estas suposiciones, junto a la consideración de conectividad, implican que este grafo sea un árbol dirigido con una sola raíz la cual no depende de ninguna otra palabra.
Esto último queda ilustrado en la figura \ref{fig:dep_tree}.
A esta estructura se le denomina \textbf{árbol de dependencias}.

\begin{figure}[h!]
	\centering
	\includegraphics[width=0.7\linewidth]{Graphics/dep_tree.png}
	\caption{Estructura de dependencias para una oración en idioma inglés y árbol de dependencias.}\label{fig:dep_tree}
\end{figure}

Existen varias restricciones adicionales que se definen sobre estas estructuras y que son más debatidas.
Una de las más conocidas es la restricción de \textbf{proyectividad}~\cite{hays1964dependency,lecerf1960programme,marcus1965notion}.
Un grafo de dependencias satisface la restricción de proyectividad con respecto a un orden linear particular de los nodos si, por cada arco $h \rightarrow d$ y un nodo $w$, $w$ ocurre entre $h$ y $d$ en el orden lineal solo si $w$ está \textbf{dominado} por $h$~\footnote{La relación \textbf{dominar} es la clausura reflexiva y transitiva de la relación de dependencia definida por los arcos}. 

\subsection{Algoritmos para el análisis de dependencias}

Con basamento en estructuras de dependencias, disímiles algoritmos han sido propuestos para analizar el lenguaje natural.
De manera general pueden distinguirse dos enfoques diferentes que se han adoptado en la literatura: análisis orientado a gramáticas y análisis orientado a datos. 

Los trabajos pioneros en el análisis orientado a gramáticas se remontan a las propuestas de \textit{Hays} y \textit{Gaifman}, cuando en 1964 y 1965 respectivamente~\cite{hays1964dependency,gaifman1965dependency}, definieron un conjunto de reglas sobre las gramáticas de dependencias y un conjunto de condiciones que debían cumplir las relaciones de dependencia.
Por su parte, los primeros intentos en realizar análisis de dependencias orientado a datos por \textit{Carroll y Charniak} en 1992~\cite{carroll1992two}, pueden considerarse también orientados a gramáticas en el sentido de que apoyaron en el formalismo de una gramática de dependencias y usaron un \emph{corpus} de datos para inducir un modelo probabilístico para la desambiguación.
En esencia, usaron una Gramática Libre del Contexto Probabilística~\cite{chomsky1956three}, que estaba restringida para ser equivalente al tipo de gramáticas de \textit{Hays} y \textit{Gaifman}. 

Muchos otros trabajos se registran en la literatura relativos a propuestas de algoritmos para el análisis de dependencias~\cite{koo2008simple,mcdonald2005non,nivre2003efficient,nivre2007maltparser,socher2011parsing}.
La propuesta más común se basa en algún tipo de algoritmo de programación dinámica con o sin desambiguación estadística.


\section{Modelo de Aprendizaje Profundo}\label{sec:model}

El modelo de aprendizaje profundo propuesto se apoya en el uso de RNN sobre estructuras derivadas del árbol de dependencias de la oración de entrada y las entidades señaladas, para obtener una representación de la supuesta relación existente entre ellas.

\subsection{Hipótesis del Camino en el Árbol de Dependencias}

Como fue explicado en el capítulo~\ref{chapter:information_extraction}, la información más completa para resolver el problema de la extracción de relaciones se encuentra en la oración completa. Sin embargo, se maneja por muchos autores la suposición de que el árbol de dependencias de la oración de entrada condensa la información vital para resolver el problema, a la vez que desecha otras fuentes de desinformación.

El hecho de que las entidades entre las cuales se quiere determinar la existencia de una relación estén formadas por múltiples palabras, constituye un problema para la configuración de algoritmo de aprendizaje profundo.
Es factible, en términos de implementación, concebir una configuración que permita establecer la existencia de relaciones entre dos palabras o \textit{tokens} de la oración.
Pero si las entidades están formadas por múltiples palabras, un aspecto cardinal de la implementación consiste en definir qué palabra va a \textit{representar} a la entidad a la hora de determinar las relaciones en las que participa.
Con este fin se han explorado diversos criterios en la literatura.
Se debate el uso de la última palabra de la entidad en algunos casos; en otros, de la primera.
De manera general, de acuerdo a observaciones realizadas sobre el conjunto de datos estudiado, las entidades señaladas constituyen unidades sintácticas coherentes; llegando a ser sintagmas nominales completos, en varios ejemplos.
Como fue abordado, uno de los criterios que determina la existencia de una dependencia con una cabecera \textit{H} en una construcción sintáctica \textit{C}, es que \textit{H} puede sustituir sintácticamente a \textit{C}.
Incluso, \textit{H} puede determinar semánticamente a \textit{C}.
Por otro lado, las entidades formadas por múltiples palabras suelen ocurrir de manera total en un subárbol del árbol de dependencias, que tiene como raíz una de las palabras de la entidad.
Este hecho fue explorado en el conjunto de datos estudiado, y se cumple en la amplia mayoría de los casos, exceptuándose muy pocos ejemplos.
Dicho subárbol correspondiente a una entidad \textit{e} lo definimos como \textbf{subárbol relevante a \textit{e}}, y lo denotamos como $S_e$.
A la raíz de dicho subárbol la denominamos \textbf{núcleo de la entidad $e$}, y se denota como $n_e$.

Otra definición importante, citada repetidamente en la literatura para resolver este problema es el \textbf{camino en el árbol de dependencias} entre dos \textit{tokens} $t_1,t_2$.
Esta definición es intuitiva, y denota a la secuencia de \textit{tokens} que ocurren en el camino que va desde $t_1$ hasta $t_2$.
Dicho camino será denotado como $C(t_1,t_2)$.

La figura \ref{fig:dep_tree_ex} ilustra las definiciones anteriores.

\begin{figure}[h!]
	\centering
	\includegraphics[width=0.7\linewidth]{Graphics/dep_tree_ex.png}
	\caption{Árbol de dependencias de la oración: \textbf{El cáncer de pulmón puede causar muerte prematura}. Se establece la relación \textbf{entails} entre las entidades \textbf{cáncer de pulmón}($e_1$) y \textbf{muerte}($e_2$). Aparecen señalados $S_{e_1}$, $S_{e_2}$, así como $C(n_{e_1}, n_{e_2})$}\label{fig:dep_tree_ex}
\end{figure}

En la selección del modelo propuesto se consideran dos hipótesis fundamentales a la hora de establecer la existencia de una relación entre dos entidades $e_1,e_2$, las cuales se comprobaron de manera experimental, como será descrito en el capítulo \ref{chapter:experiments}:

\begin{enumerate}
	\item $S_{e_1}$ y $S_{e_2}$ condensan información útil sobre las respectivas entidades, a la vez que deshechan fuentes de desinformación.
	
	\item $C(n_{e_1}, n_{e_2})$ condensa información útil para determinar una posible relación entre $e_1$ y $e_2$, a la vez que deshecha fuentes de desinformación.
\end{enumerate}

\subsection{Red Neuronal}

Dada una oración de entrada, se utiliza un vector $w_i$ para representar la palabra $i-$ésima de la misma. Dicho vector se obtiene a partir de la concatenación de \textit{embeddings} provenientes de distintas fuentes de información:

\begin{description}
	\item[Palabras:] Se utilizan \textit{embeddings} preentrenados en un corpus construido a partir de artículos de Wikipedia con contenido médico.
	Fueron entrenados utilizado el algoritmo \textbf{word2vec}\cite{word2vec} con la arquitectura \textbf{skipgram}.
	
	\item[Caracteres:] Se utilizan \textit{embeddings} obtenidos mediante una CNN sobre los caracteres de la palabra.
	
	\item[POST-tag y Dependencia:] Se utiliza un \textit{embedding} de la etiqueta de POS-tag de la palabra y la dependencia de la misma con su ancestro en el árbol de dependencias de la oración.
	
	\item[Etiqueta BMEWO-V y Tipo de la entidad:] Se añaden \textit{embeddings} con información relativa a la entidad a la que pertenece la palabra en cuestión.
	En este caso la etiqueta correspondiente en el sistema BMEWO-V así como el tipo de la entidad.
	
\end{description}

Sean además, $p_1, p_2,\dots, p_k$ las posiciones de las palabras que se encuentran en el camino en el árbol de dependencias entre las dos entidades señaladas, siendo $e_1,e_2$ dichas entidades, respectivamente. Sea $W = [w_{e_1};w_{p_1};\dots;w_{p_k};w_{e_2}]$ la concatenación de los vectores en dichas posiciones. Nótese que $W$ tiene dimensión $(k+2) \times q$, siendo $q$ la cantidad de componentes de los vectores $w_i$.

Primeramente, una capa BiLSTM transforma la representación de cada palabra de la secuencia $W$, para incluir información contextual de las palabras anteriores y posteriores de cada posición:

\begin{equation*}
	P = BiLSTM(W)
\end{equation*}

Luego, la capa LSTM apilada enfatiza la direccionalidad de la relación, procesando la secuencia $P$ solamente en el sentido que va desde el origen hacia el destino de la supuesta relación, y obtiene un vector $p$ que codifica toda la información contenida en el camino del árbol de dependencias.

\begin{equation*}
p = LSTM(P)
\end{equation*}

Entre tanto, una capa recurrente se aplica sobre el subárbol que tiene como raíz a cada una de las entidades señaladas.
En este caso se utiliza una red Tree-LSTM, en su variante $Child Sum$ \cite{treeLSTM}.
Siendo $T_{e_1}$ y $T_{e_2}$ dichos subárboles:

\begin{equation*}
	t_{e_1} = TreeLSTM(T_{e_1})
\end{equation*}


\begin{equation*}
	t_{e_2} = TreeLSTM(T_{e_2})
\end{equation*}


Una red Tree-LSTM es una generalización de una red LSTM, que permite procesar de manera recurrente estructuras de vectores que se organicen en forma de grafos dirigidos y acíclicos~(como los árboles, por ejemplo). 

Una celda Tree-LSTM en el momento $t$, al igual que su homólogo lineal, se define como una colección de vectores en $\mathbb{R}^d$, siendo $d$ la dimensión oculta de dicha celda.
Estos vectores son: una compuerta de entrada $i_t$, una compuerta de olvido $f_t$, una compuerta de salida $o_t$, una celda de memoria $c_t$ y un estado oculto $h_t$. Siendo $C(j)$ la secuencia de hijos de el nodo $j$, las ecuaciones de transición de una celda Tree-LSTM se muestran a continuación:

\begin{equation*}
	\overline{h}_j = \sum_{k\in C(j)} h_k
\end{equation*}

\begin{equation*}
i_j = \sigma(W^{(i)}x_j + U^{(i)}\overline{h}_j + b(i))
\end{equation*}

\begin{equation*}
f_{jk} = \sigma(W^{(f)}x_j + U^{(f)}h_k + b(f))
\end{equation*}

\begin{equation*}
o_j = \sigma(W^{(o)}x_j + U^{(o)}\overline{h}_j + b(o))
\end{equation*}

\begin{equation*}
u_j = tanh(W^{(u)}x_j + U^{(u)}\overline{h}_j + b(u))
\end{equation*}

\begin{equation*}
c_j = i_j \odot u_j + \sum_{k\in C(j)} f_{jk} \odot c_k
\end{equation*}

\begin{equation*}
h_j = o_j \odot tanh(c_j)
\end{equation*}

Nótese que la diferencia con respecto a las redes LSTM radica en que, como se tienen varios momentos inmediatos anteriores al momento $t$, se considera la suma de todos los respectivos estados ocultos.

Loa vectores obtenidos a partir de la secuencia de entrada y las entidades señaladas se concatenan para formar la representación de la hipotética relación.

\begin{equation*}
	r = [t_{e_1};t_{e_2}, p]
\end{equation*}

La salida $o$ se obtiene a partir de aplicar la función sigmoide a una transformación linear de dicho vector.

\begin{equation*}
	o = \sigma(W^{(o)}r + b^{(o)})
\end{equation*}

La matriz $W^{(o)}$ tiene dimensiones $m \times l$, donde $l = |t_{e_1}| + |t_{e_2}| + |p|$ y $m$ es el número de relaciones semánticas diferentes que se definen.

De acuerdo al vector de salida $o$, se predice la existencia de una relación si su valor máximo excede un umbral prefijado que se introduce como un hiperparámetro adicional. De ser así, se predice la existencia solamente de la relación dada por $\arg\max(o)$.
	
Nótese como, a diferencia de las arquitecturas tradicionales, esta configuración permite prescindir de la relación ficticia \textit{none}.

La figura \ref{fig:rel_model} ilustra la arquitectura descrita.

\begin{figure}[h!]
	\centering
	\includegraphics[width=1\linewidth]{Graphics/rel_model_class.jpg}
	\caption{Arquitectura de red utilizada. La oración de entrada es \textit{El cáncer de pulmón puede causar muerte prematura} Y las entidades en cuestión son \textit{cáncer de pulmón} y \textit{muerte}.}\label{fig:rel_model}
\end{figure}

\section{Entrenamiento}

%===================================================================================
%===================================================================================
% Chapter: Análisis Experimental
%===================================================================================
\chapter{Análisis Experimental}\label{chapter:experiments}
\addcontentsline{toc}{chapter}{Análisis Experimental}

Este capítulo se centra en la descripción del los detalles de la implementación de las propuestas descritas para la extracción de entidades y relaciones.
Se explican las configuraciones de los exprimentos realizados y el conjunto de técnicas experimentales empleadas, se muestran los resultados de dicho estudio y se someten los mismo a una posterior discusión.

La implementación de las propuestas se realizó utilizando el lenguaje de programación Python.
Se utilizó la biblioteca \textbf{PyTorch} como marco para el trabajo con redes neuronales profundas.
Para entrenar los modelos y llevar a cabo los distintos exprimentos se utilizaron los conjuntos de datos correspondientes a los eventos eHealth-KD en sus ediciones del 2019~\footnote{repo github} y 2020~\footnote{repo github}~\footnote{Se especificará adecuadamente que datos fueron utilizados en cada uno de los experimentos}.

\section{Marco Experimental}

Los algoritmos definidos para la resolución de ambas tareas, están basados en técnicas de aprendizaje profundo.
Una de las implicaciones de esta decisión, es que una vez fijo el algoritmo, existe una amplia variedad de hiperparámetros que se pueden ajustar en virtud de obtener mejores resultados computacionales.
Los experimentos realizados toman como referencia la configuración de hiperparámetros que mejores resultados alcanzó.

Para la evaluación definitiva de las propuestas se utilizaron métricas de tipo F1, que fueron definidas para la evaluación de las propuestas en la competencia eHealth-KD 2019~\cite{ehalth19}.
La precisión y el recobrado se definen en términos de \textit{Correct}(\textbf{C}), \textit{Missing}(\textbf{M}), \textit{Spurious}(\textbf{S}), \textit{Incorrect}(\textbf{I}) y \textit{Partial}\textbf{(P)}.

Una entidad es clasificada como \textit{Correct}, si coincide con alguna de las entidades de la oración en cuanto a las palabras que contiene y la etiqueta asignada a la misma.
Si las palabras coinciden pero la etiqueta no es correcta se clasifica como \textit{Incorrect}.
Si existe un solapamiento entre una entidad extraída y una presente en la oración con la misma etiqueta, cuenta como una ocurrencia \textit{Partial}.
Si en una entidad extraída no ocurre ninguno de los casos anteriores se considera \textit{Spurious}.
Por su parte, una entidad presente en la oración y no detectada por el modelo se considera \textit{Missing}.

:as relaciones solo se clasifican en términos de \textit{Correct}, \textit{Spurious} y \textit{Missing}, teniendo en cuenta las entidades que enlaza y la etiqueta asignada a la misma.

La fórmula \ref{equation:f1_formula} define la métrica F1 utilizada.
Los subíndices $A$ y $B$ identifican a las tareas de extracción de entidades y relaciones, respectivamente.

\begin{equation*}
R_{sc1} = \frac{C_A + \frac{1}{2}P_A + C_B}{C_A + I_A + P_A + M_A + C_B + M_B}
\end{equation*}

\begin{equation*}
P_{sc1} = \frac{C_A + \frac{1}{2}P_A + C_B}{C_A + I_A + P_A + S_A + C_B + S_B}
\end{equation*}

\begin{equation}
F1_{sc1} = 2\frac{P_{sc1}R_{sc1}}{P_{sc1}+R_{sc1}}
\end{equation}\label{equation:f1_formula}

Los escenarios 2 y 3 se evaluaron de manera equivalente utilizando las fórmulas correspondientes.

Se recogen en este estudio distintas configuraciones de hiperparámetros exploradas, así como detalles del proceso de entrenamiento de las propuestas con mejores resultados.
Se analizaron las curvas de aprendizaje para medir el impacto que tiene en la efectividad el tamaño del conjunto de datos utilizado.
Con el objetivo de medir la influencia de las representaciones distribuidas empleadas, se realizó un análisis ablasivo sobre las distintas componentes de la entrada de los modelos.

En el caso del modelo para la extracción de relaciones, se evaluó adicionalmente las hipótesis sobre el árbol de dependencias.
Para ello se entrenó un modelo con una complejidad semejante en términos de parámetros.
Se sustiuyó el camino en el árbol de dependencias por la secuencia de palabras de la oración completa, y se cambió el procesamiento mediante Tree-LSTM del subárbol relevante a las entidades, por una codificación basada en BiLSTM de las palabras que forman parte de las mismas.


\section{Resultados Computacionales}

\section{Discusión}
%===================================================================================

\backmatter

%===================================================================================
% Chapter: Conclusiones
%===================================================================================
\chapter*{Conclusiones}\label{chapter:conclusions}
\addcontentsline{toc}{chapter}{Conclusiones}

%===================================================================================
\include{BackMatter/Bibliography}

\end{document}