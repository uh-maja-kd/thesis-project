%===================================================================================
% Chapter: Conclusiones
%===================================================================================
\chapter*{Conclusiones}\label{chapter:conclusions}
\addcontentsline{toc}{chapter}{Conclusiones}

Este trabajo propone un sistema basado en aprendizaje profundo haciendo uso de t\'ecnicas presentes en el estado del arte para resolver las Tareas A y B de los eventos (PONER EVENTOS), relacionados con los problemas de \textbf{NER} y \textbf{RE}. El sistema propuesto se entrena y eval\'ua dentro del dominio de la salud, pero es genralizable a otros dominios. Entre las contribuciones fundamentales del trabjo destacan: (1) la implementaci\'on de un sistema de extracci\'on autom\'atica de las anotaciones desde oraciones en lenguaje natural; (2) el dise\~no de modelos para resolver las Tareas A y B respectivamente con resultados competitivos y similares al estado del arte; (3) el an\'alisis del impacto de algunas representaciones de la entrada, as\'i como de las componentes de la arquitectura de cada modelo en los resultados finales. 

El modelo presentado para resolver la Tarea A hace uso de 
%===================================================================================